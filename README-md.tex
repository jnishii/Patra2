\documentclass[]{jsarticle}
\usepackage[hypertex]{hyperref}
\usepackage{amsmath}
\usepackage{ascmac}
%\usepackage[height=24cm,width=16cm]{geometry}
\usepackage{fancyhdr}

\newcommand{\passthrough}[1]{#1}
%\newcommand{\passthrough}[1]{\lstset{mathescape=false}#1\lstset{mathescape=true}}
\renewcommand{\labelenumi}{\arabic{enumi})}
\renewcommand{\labelenumii}{\alph{enumii})}

\providecommand{\tightlist}{%
   \setlength{\itemsep}{0pt}\setlength{\parskip}{0pt}}

\begin{document}


\noindent
%\rule{\linewidth}{0.3pt}
%\rule{\linewidth}{0.3pt}
  

\hypertarget{patra2ux306eux30a4ux30f3ux30b9ux30c8ux30fcux30eb}{%
\section{Patra2のインストール}\label{patra2ux306eux30a4ux30f3ux30b9ux30c8ux30fcux30eb}}

ゾウリムシの追跡プログラム

\begin{itemize}
\tightlist
\item
  https://github.com/bcl-group/motion-capture からの派生
\item
  上記motion-captureは\href{https://qiita.com/hitomatagi/items/a4ecf7babdbe710208ae}{OpenCVを使ったモーション
  テンプレート解析}を参考に作成
\end{itemize}

\hypertarget{ux74b0ux5883ux69cbux7bc9}{%
\subsection{環境構築}\label{ux74b0ux5883ux69cbux7bc9}}

\begin{enumerate}
\tightlist
\item
  Python環境のインストール
\end{enumerate}

\begin{lstlisting}
$ pyenv install 3.10.2
$ pyenv rehash
$ pyenv global 3.10.2 # インストールしたPython 3.10.2 を利用するための設定
\end{lstlisting}

\begin{enumerate}
\setcounter{enumi}{1}
\tightlist
\item
  確認
\end{enumerate}

\begin{lstlisting}
$ pyenv versions
    system
* 3.10.2 (set by /usr/local/var/pyenv/version)
$ which python
  /usr/local/var/pyenv/version/shims/python
$ python --version
Python 3.10.2
\end{lstlisting}

\begin{enumerate}
\setcounter{enumi}{2}
\tightlist
\item
  \href{https://python-poetry.org/docs/}{Poetry}のインストール
\end{enumerate}

PoetryはPythonのライブラリ管理ツール。これを使ってPatra2に必要なPythonライブラリを管理する。

\begin{lstlisting}
$ curl -sSL https://install.python-poetry.org | python3 -
\end{lstlisting}

\begin{enumerate}
\setcounter{enumi}{3}
\tightlist
\item
  Patraのダウンロード
\end{enumerate}

\begin{lstlisting}
git clone git@github.com:jnishii/Patra2.git
\end{lstlisting}

\begin{enumerate}
\setcounter{enumi}{4}
\tightlist
\item
  必要なライブラリのインストール
\end{enumerate}

\begin{lstlisting}
$ cd Patra2
$ poetry install
\end{lstlisting}

\hypertarget{patra2ux306eux4f7fux3044ux65b9}{%
\subsection{Patra2の使い方}\label{patra2ux306eux4f7fux3044ux65b9}}

\hypertarget{ux4f7fux7528ux4e0aux306eux6ce8ux610f}{%
\subsubsection{使用上の注意}\label{ux4f7fux7528ux4e0aux306eux6ce8ux610f}}

\begin{itemize}
\tightlist
\item
  解析したいファイルは patara2/Data/
  の下に置く。サブフォルダを作っても良い。
\item
  出力ファイルは patra2/Data/
  の下に作られるサブフォルダ(ファイル名と同じ名前)に保存される。
\end{itemize}

\hypertarget{ux89e3ux6790ux306eux9806ux756a}{%
\subsubsection{解析の順番}\label{ux89e3ux6790ux306eux9806ux756a}}

\begin{itemize}
\tightlist
\item
  キャリブレーション(calibration.py)
\item
  ゾウリムシの追跡 \& 行動解析 (Patra2.py)
\item
  速さのフーリエ解析(Speed\_fourier.py)
\end{itemize}

\hypertarget{ux30adux30e3ux30eaux30d6ux30ecux30fcux30b7ux30e7ux30f3ux306eux53d6ux308aux65b9-calibration.py}{%
\subsubsection{キャリブレーションの取り方
(calibration.py)}\label{ux30adux30e3ux30eaux30d6ux30ecux30fcux30b7ux30e7ux30f3ux306eux53d6ux308aux65b9-calibration.py}}

\begin{enumerate}
\tightlist
\item
  マイクロメータの動画ファイルを解析ファイルと同様に460×640ピクセルとし、10mm×10mmのマイクロメータが収まるようにを撮影しておくこと!!!
\item
  \passthrough{\lstinline!calibration.py!}を起動し,キャリブレーション用ファイルを読み込む
\end{enumerate}

\begin{lstlisting}
poetry run python calibration.py <マイクロメータの動画ファイル名>
\end{lstlisting}

実行例

\begin{lstlisting}
poetry run python calibration.py Data/chiba/1111/calibration.mov
\end{lstlisting}

\begin{enumerate}
\setcounter{enumi}{2}
\tightlist
\item
  5mm × 5mm(1目盛り50μm)
  のマイクロメータの目盛り線を、縦目盛りの5mmの線上 →
  縦目盛りの0mmの線上 → 横目盛りの0mmの線上 → 横目盛りの5mmの線上
  の順にクリック
\item
  実行するとターミナルに以下のような出力が出る。単位はμm/pixel
\end{enumerate}

\begin{lstlisting}
X_scale: 15.1515151515 
Y_scale: 15.5279503106 
\end{lstlisting}

\hypertarget{ux30beux30a6ux30eaux30e0ux30b7ux306eux8ffdux8de1-ux884cux52d5ux89e3ux6790-patra2.py}{%
\subsubsection{ゾウリムシの追跡 \& 行動解析
(Patra2.py)}\label{ux30beux30a6ux30eaux30e0ux30b7ux306eux8ffdux8de1-ux884cux52d5ux89e3ux6790-patra2.py}}

\hypertarget{ux5b9fux884cux65b9ux6cd5}{%
\paragraph{実行方法}\label{ux5b9fux884cux65b9ux6cd5}}

\textbf{実行方法その1}

Patra2のディレクトリ内で以下を実行

\begin{lstlisting}
$ poetry run python Patra2.py 解析したいディレクトリ/解析したい動画ファイル X_scaleの値 Y_scaleの値
\end{lstlisting}

\textbf{例}

\begin{lstlisting}
$ poetry run python Patra2.py Data/chiba/1111/2836_001.mov 17.161340059327287 16.849643261459075
\end{lstlisting}

\textbf{実行方法その2}

\passthrough{\lstinline!run.sh!}の中に,上記コマンドが書いてある。エディタで修正して以下を実行

\begin{lstlisting}
$ ./run.sh
\end{lstlisting}

\hypertarget{ux51faux529bux30c7ux30fcux30bf}{%
\paragraph{出力データ}\label{ux51faux529bux30c7ux30fcux30bf}}

\passthrough{\lstinline!outputs/!}の下に解析した動画ファイル名のフォルダができる。

\begin{itemize}
\tightlist
\item
  ローパス(バターワース)処理: 直流成分は1/2,
  約0-7.4Hzまではほぼ100\%透過, 減衰域は7.4925Hz-9.99Hz
\item
  軌道データファイル \passthrough{\lstinline!Para<id>\_<filename>.csv!}

  \begin{itemize}
  \tightlist
  \item
    データ列:
    \passthrough{\lstinline!time,para\_id,X,Y,X\_lpf,Y\_lpf,Vx,Vy,Vx\_lpf,Vy\_lpf,V,V\_lpf,V\_lpf2!}
  \item
    time: 時間
  \item
    para\_id: ゾウリムシのID
  \item
    X,Y: 位置情報(μm)
  \item
    X\_lpf, Y\_lpf: X,Yをローパス処理したもの
  \item
    Vx, Vy: X,Yの差分で求めた速度(μm/s)
  \item
    V: Vx,Vyから求めた速さ
  \item
    Vx\_lpf, Vy\_lpf: X\_lpf, Y\_lpfから求めた速度
  \item
    V\_lpf: Vx\_lpf, Vy\_lpfから求めた速度
  \item
    V\_lpf2: V\_lpfをローパス処理したもの (\textbf{注意}:
    稀に負の値になることがある。
  \end{itemize}
\item
  周波数分析データファイル
  \passthrough{\lstinline!Para<id>\_<filename>\_freq.csv!}

  \begin{itemize}
  \tightlist
  \item
    データ列:
    \passthrough{\lstinline!,X\_freq,X\_amp,Y\_freq,Y\_amp,V\_lpf2\_freq,V\_lpf2\_amp!}
  \item
    X\_freq,X\_amp,Y\_freq,Y\_amp: X, Y軌道の周波数分析結果
  \item
    V\_lpf2\_freq,V\_lpf2\_amp: V\_lpf2の周波数分析結果
  \end{itemize}
\item
  周波数ピーク分析データファイル

  \begin{itemize}
  \tightlist
  \item
    上記周波数分析データファイルからピーク周波数を抽出したもの
  \item
    \passthrough{\lstinline!Para<id>\_<filename>\_X\_peak\_freq.csv!}
  \item
    \passthrough{\lstinline!Para<id>\_<filename>\_Y\_peak\_freq.csv!}
  \item
    \passthrough{\lstinline!Para<id>\_<filename>\_V\_peak\_freq.csv!}
  \end{itemize}
\end{itemize}

\hypertarget{ux305dux306eux307bux304b}{%
\subsubsection{そのほか}\label{ux305dux306eux307bux304b}}

\begin{itemize}
\tightlist
\item
  画像の出力形式はpngで良い?
\item
  グラフはまとめたほうが良い? ばらばらが良い?
\end{itemize}

\end{document}
